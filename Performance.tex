\section{Software performance}
Characterization of software performance is a complex problem, that
can be approached from different points of view. While software
developers are most concerned on understanding which parts of the code
need optimizing, from the point of view of the computing
infrastructure manager it is mainly a question of measuring what is
needed to run effectively the experiment workloads. In this case, the
application software is to be considered, at a first approximation,
like a ``black box'', and tools like PrMon (which relies on the Linux
kernel to extract CPU time, memory, I/O and network metrics for a
given process tree)~\cite{prmon} or Trident (which gives access to
detailed information on the CPU utilization at the node level using
hardware counters)~\cite{trident} are extremely effective in producing
metrics that can be used for infrastructure planning, for benchmarking
and for understanding inefficiencies.

A set of reference workloads from each LHC experiment was analyzed
with PrMon, and the resulting values for the metric are summarized in
table~\ref{table:prmon}. The metrics include: number of threads or
processes ($N_{thr/proc}$), CPU efficiency ($\epsilon_{CPU}$), time per
event ($T_{evt}$), memory per core ($M_c$), read rate per core ($R_c$)
and write rate per core ($W_c$).

\begin{table}
\centering
\caption{Metrics measured by PrMon on a set of reference workloads with an Intel Xeon E5-2630. Values are approximate and may change with different versions of the experiment software}
\label{table:prmon}
\begin{tabular}{lrrrrrrrr}
\hline
job & $N_{thr/proc}$ & $\epsilon_{CPU}$ & $T_{evt}$ (s) & $M_{c}$ (GB) & $R_{c}$ (MB/s) & $W_{c}$ (MB/s) \\\hline
ALICE sim & 1 & 100\% & 10.9 & 0.96 & 0.08 & 0.17 \\
ATLAS G4 & 8 & 100\% & 270 & 0.44 & 0.015 & 0.009 \\
ATLAS digireco & 8 & 87\% & 56 & 1.1 & 0.3 & 0.24 \\
ATLAS deriv & 8 & 98\% & 0.7 & 1.2 & 0.7 & 0.07 \\
CMS gensim & 8 & 99\% & 21 & 0.2 & 0.05 & 0.04 \\
CMS digi & 8 & 78\% & 5.9 & 0.6 & 0.3 & 0.3 \\
CMS reco & 8 & 83\% & 9.8 & 0.45 & 0.3 & 0.2 \\
LHCb gensim & 1 & 100\% & 180 & 0.9 & 0.3 & 0.01 \\\hline
\end{tabular}
\end{table}
As the full PrMon output consists of time series for each metric, we
looked into ways to parametrize the time series using a minimal set of
parameters. A technique based on CPOP (Continuous piecewise linear
Pruned Optimal Partitioning)~\cite{cpop} is able to detect
changepoints and therefore reduce a time series to a very small number
of points. Another work looked into the effect of varying limitations
of system resources (memory, network bandwidth and network latency) on
the reference workloads, in particular on their wall-clock time. It is
planned to combine the two anlyses and study the results of the
parametrisation as functions of the resource restrictions; this would
allow to have a very simple input to a future model of large scale
workloads running on a computing infrastructure, like a general
purpose batch cluster or an HPC resource.

Studies on the effect of compiler versions and optimisations were done
using Geant4 simulation, which showed that statically compiling
libraries may achieve a 10\% speedup with respect to dynamically
compiled libraries, and switching from gcc 4.8.5 to 8.2.0 resulted in
a 30\% speedup~\cite{marcon}. Consistent results were obtained for CMS
simulation~\cite{alejandro}.
