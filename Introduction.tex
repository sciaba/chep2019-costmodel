\section{Introduction}
The preparation for the LHC Run 3, which will see considerable changes in
data collection and processing for ALICE and LHCb, and for Run 4, or HL-LHC,
has made significant progress in the last year, and many factors have
contributed to decrease the estimated gap between the amount of processing
power and storage estimated to be available, and those estimated to be needed.
In the previous report~\cite{costmodel} we quoted a $O(10)$ discrepancy,
while now it is closer to a factor 2-3. The ``revolutionary'' changes we
mentioned as being required to completely close the gap are being in fact
introduced or planned, as we will see in this contribution.

In addition, several refinements have been made in the calculation of
the cost of computing, both in terms of resources with respect to what is
required for the physics programme, and in terms of infrastructure costs.

The System Performance and Cost Modeling working group, created in 2017
and comprising around thirty members from experiments, sites and IT and
software experts, has continued along the roadmap initially defined and
added some new activities, mostly in the area of data access efficiency,
in close collaboration with the DOMA access group~\cite{domaaccess}.

