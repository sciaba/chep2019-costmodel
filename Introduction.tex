\section{Introduction}
The preparation for the LHC Run 3, which will see considerable changes
in data collection and processing for ALICE and LHCb, and for Run 4,
or HL-LHC, has made significant progress in the last year; many
factors have contributed to decrease the estimated gap between the
estimated amounts of available and needed processing power and
storage.  In the previous report~\cite{costmodel} we quoted a $O(10)$
discrepancy, while now it is closer to a factor 2-3~\cite{extrap}. The
``revolutionary'' changes we mentioned as being required to completely
close the gap are progressively being introduced or planned.

In addition, thanks to several refinements, the calculation of the
cost of computing has improved, both in terms of resources for what is
required for the physics program, and in terms of infrastructure
costs.

The System Performance and Cost Modeling Working Group, created in
2017 and comprising around thirty members from experiments, sites, IT
and software experts, has continued along the roadmap initially
defined and even started some new activities, as in the area of data
access efficiency, in close collaboration with the DOMA access
group~\cite{domaaccess} and in contributing to the definition of new
benchmarks, together with the HEPiX benchmarking working
group~\cite{bench}.

In this contribution we will show some recent developments in the
areas of work under the domain of this activity.
