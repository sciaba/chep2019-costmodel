\section{Conclusions}
After two years of activity it is the time to re-examine the roadmap
and the goals of the working group. Many of its activities have
matured and would be best conducted in other groups, among which DOMA
access for the storage cost efficiency studies and the HEP Software
Foundation or the experiments for performance analysis tools and
compiler optimization studies.  Resource estimation falls solidly in
the domain of experiments, while site cost estimation is still ideally
covered by the cost model working group.

Perhaps one of the biggest achievements was to raise the attention and
stimulate the community to think, more then even before, on how to
overcome the capacity gap that would make HL-LHC computing impossible,
if not addressed. This was possible also by participating to computing
schools or contributing to workshops related to efficiency and
performance of HEP software.

Therefore it seems appropriate to re-scope the working group by
concentrating on topics unique to it (site cost calculation, analysis
of cost differences) and run topical workshops with experiments and
sites rather than having regular meetings.
